%-------------------------
% Resume in Latex
% Author : Sourabh Bajaj
% Modified by: Junqi Zhang
% License : MIT
%------------------------

\documentclass[letterpaper,12pt]{article}

\usepackage{latexsym}
\usepackage[empty]{fullpage}
\usepackage{titlesec}
\usepackage{marvosym}
\usepackage[usenames,dvipsnames]{color}
\usepackage{verbatim}
\usepackage{enumitem}
\usepackage[hidelinks]{hyperref}
\usepackage{fancyhdr}
\usepackage[english]{babel}
\usepackage{tabularx}
\usepackage{fontawesome5}
\usepackage{multicol}
\setlength{\multicolsep}{-3.0pt}
\setlength{\columnsep}{-1pt}

% macOS style font (Inter is very close to San Francisco)
\usepackage[sfdefault]{inter}
\usepackage[T1]{fontenc}
\usepackage{microtype}

% Adjust line spacing
\linespread{1.1}

\input{glyphtounicode}

\pagestyle{fancy}
\fancyhf{} % clear all header and footer fields
\fancyfoot{}
\renewcommand{\headrulewidth}{0pt}
\renewcommand{\footrulewidth}{0pt}
\setlength{\footskip}{5pt}

% Adjust margins
\addtolength{\oddsidemargin}{-0.5in}
\addtolength{\evensidemargin}{-0.5in}
\addtolength{\textwidth}{1in}
\addtolength{\topmargin}{-0.5in}
\addtolength{\textheight}{1.0in}

\urlstyle{same}

\raggedbottom
\raggedright
\setlength{\tabcolsep}{0in}

% Sections formatting
\titleformat{\section}{
  \vspace{-4pt}\bfseries\raggedright\large\color{RoyalBlue}
}{}{0em}{}[\color{RoyalBlue}\titlerule \vspace{-5pt}]

% Custom commands
\newcommand{\resumeItem}[1]{
  \item\small{
    {#1 \vspace{-2pt}}
  }
}

\newcommand{\resumeSubheading}[4]{
  \vspace{-2pt}\item
    \begin{tabular*}{0.97\textwidth}[t]{l@{\extracolsep{\fill}}r}
      \textbf{#1} & #2 \\
      \textit{\small#3} & \textit{\small #4} \\
    \end{tabular*}\vspace{-7pt}
}

\newcommand{\resumeSubSubheading}[2]{
    \item
    \begin{tabular*}{0.97\textwidth}{l@{\extracolsep{\fill}}r}
      \textit{\small#1} & \textit{\small #2} \\
    \end{tabular*}\vspace{-7pt}
}

\newcommand{\resumeProjectHeading}[2]{
    \item
    \begin{tabular*}{0.97\textwidth}{l@{\extracolsep{\fill}}r}
      \small#1 & #2 \\
    \end{tabular*}\vspace{-7pt}
}

\newcommand{\resumeSubItem}[1]{\resumeItem{#1}\vspace{-4pt}}

\renewcommand\labelitemii{$\vcenter{\hbox{\tiny$\bullet$}}$}

\newcommand{\resumeSubHeadingListStart}{\begin{itemize}[leftmargin=0.15in, label={}]}
\newcommand{\resumeSubHeadingListEnd}{\end{itemize}}
\newcommand{\resumeItemListStart}{\begin{itemize}}
\newcommand{\resumeItemListEnd}{\end{itemize}\vspace{-5pt}}

%-------------------------------------------
%%%%%%  CV STARTS HERE  %%%%%%%%%%%%%%%%%%%%%%%%%%%%


\begin{document}

%----------HEADING-----------------
\begin{center}
    {\Huge \textbf{\color{RoyalBlue} Junqi Zhang}} \\ \vspace{5pt}
    \small \href{mailto:jzhang2648@wisc.edu}{\raisebox{-0.2\height}\faEnvelope\  \underline{jzhang2648@wisc.edu}} ~ 
    \href{https://john137fs.github.io/jqzhang.github.io/index.html}{\raisebox{-0.2\height}\faGlobe\ \underline{john137fs.github.io/jqzhang.github.io}}
    \vspace{-8pt}
\end{center}


%-----------EDUCATION-----------------
\section{Education}
  \resumeSubHeadingListStart
    \resumeSubheading
      {University of Wisconsin - Madison}{Madison, WI}
      {PhD in Physics}{Started Aug. 2025}
    \resumeSubheading
      {University of Wisconsin - Madison}{Madison, WI}
      {BS in Physics and Philosophy}{Jan. 2023 -- May. 2025}
  \resumeSubHeadingListEnd

%-----------Interest-----------------
\section{Research Interest}
  \resumeSubHeadingListStart
    \resumeItem{Early universe cosmology, particularly inflationary physics, reheating and associated observables.}
  \resumeSubHeadingListEnd

%-----------EXPERIENCE-----------------
\section{Research Experience}
  \resumeSubHeadingListStart

    \resumeSubheading
      {Inflationary Cosmology}{Madison, WI}
      {Advisor: Prof. Daniel Chung}{Mar 2024 -- Sep 2024}
      \resumeItemListStart
        \resumeItem{\textbf{Cosmological Perturbation}: Reproduced the basic results of the linear order cosmological perturbation with multiple scalar fields under uniform curvature gauge.}
        \resumeItem{\textbf{Isocurvature Perturbation}: Investigated double monodromy inflation model and embedded the monodromy axion model into the isocurvature perturbation with realistic parameters as the dominant inflaton with a superheavy dark matter spectator. Computed spectator perturbation spectrum in the superhorizon region.}  
      \resumeItemListEnd

  \resumeSubHeadingListEnd

%-----------Teaching Experience-----------------
\section{Teaching Experience}
  \resumeSubHeadingListStart
    \resumeItem{TA for Physics 207: General Physics I (Fall. 2025)}
  \resumeSubHeadingListEnd

%-----------Skills-----------------
\section{Skills}
  \resumeSubHeadingListStart
    \resumeItem{\textbf{Languages}: Mathematica, Julia, Python}
  \resumeSubHeadingListEnd

%-------------------------------------------
\end{document}
